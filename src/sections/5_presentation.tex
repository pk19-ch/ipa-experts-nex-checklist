\section{Präsentation, Demonstration und Fachgespräch}
Beachte, dass dieser Tag gerade für den Kandidaten ein spezieller Tag ist.\\
Verleihe dem Anlass ruhig etwas Würdevolles.\\
An der Präsentation, Demonstration und Fachgespräch nehmen nur der Kandidat, die verantwortliche Fachkraft und die Experten teil.\\
Maximal eine weitere Person, welche massgeblich an der Umsetzung der Facharbeit beteiligt war, darf auf Antrag beim CEX ebenfalls an der Präsentation und der Demonstration teilnehmen.\\
Der Hauptexperte leitet durch diesen Anlass.\\
Du unterstützt den Hauptexperten aktiv mit deinen Beobachtungen und Notizen.\\
Es gelten die Regeln vom QV-Leitfaden.

\begin{taskitemwithoutcomment}{Begrüssung}
  Stelle dich und deine Rolle als NEX vor.\\
\end{taskitemwithoutcomment}  
\begin{taskitemwithoutcomment}{Gesundheit}
  Der HEX muss den Kandidaten fragen, ob er gesund und in der Lage ist, die Prüfung zu absolvieren.\\
  Falls nicht, muss der Kandidat zum Arzt und sich ein ärztliches Zeugnis ausstellen lassen.\\
  In diesem Fall wird der Termin verschoben und es muss ein neuer Termin vereinbart werden.
\end{taskitemwithoutcomment}
\begin{taskitemwithoutcomment}{Kriterienkatalog bereit halten}
  Im Kriterienkatalog findest du im Teil 3 die Leitfragen zu der Präsentation und Demonstration mit den Hinweisen, auf was du achten musst.
\end{taskitemwithoutcomment}
\begin{taskitemwithoutcomment}{Präsentation}
  Der Hauptexperte wird dem Kandidaten das Wort für den Start der Präsentation übergeben.\\
  Stelle keine Fragen während der Präsentation, auch wenn der Kandidat dir das zugestehen würde. Dazu ist später noch Zeit.
\end{taskitemwithoutcomment}
\begin{taskitem}{Beobachtungen zur Präsentation}
  Beachte die Leitfragen im Teil 3 des Kriterienkatalogs.\\
  Achte auf den Aufbau und den Inhalt der Präsentation, aber auch auf die Dauer der Präsentation, den Einsatz der vorhandenen Mittel und den Vortragsstil. Die Kandidaten haben gelernt, wie Vorträge zu halten sind.\\
  Mache dir Notizen.
\end{taskitem}
\begin{taskitemwithoutcomment}{Demonstration}
  Der Hauptexperte wird zur Demonstration überleiten.\\
  Während der Demonstration darfst du spontan Fragen stellen.\\
  Doch lass den Kandidaten den vorbereiteten Ablauf durchspielen. Es kommt ja noch das Fachgespräch. 
\end{taskitemwithoutcomment}
\begin{taskitem}{Beobachtungen zur Demonstration}
  Beachte die Leitfragen im Teil 3 des Kriterienkatalogs.\\
  Achte auf den Aufbau und den Inhalt der Demonstration.\\
  Mache dir Notizen.
\end{taskitem}
\begin{taskitemwithoutcomment}{Fachgespräch}
  Der Hauptexperte wird zum Fachgespräch überleiten.\\
  Am Fachgespräch dürfen nur die Experten, die verantwortliche Fachkraft und der Kandidat teilnehmen.\\
  Der Hauptexperte wird intensiv mit dem Gespräch beschäftigt sein. Deshalb sind hier deine Notizen zu den Beobachtungen und Antworten des Kandidaten besonders wichtig.
\end{taskitemwithoutcomment}
\begin{taskitemwithoutcomment}{Beobachtungen zu den Fachfragen}
  Nimm die vom Hauptexperten bereitgestellten Fachfragen zur Hand.\\
  Mache die Notizen zu den Antworten des Kandidaten direkt in das Formular für die Fachfragen.
\end{taskitemwithoutcomment}
