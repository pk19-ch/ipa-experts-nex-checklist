\section{Von der Abgabe bis zur Präsentation}
Wenn die 10 Tage der IPA vorbei sind und der Kandidat den Bericht hochgeladen hat, dann nimm dir Zeit, den IPA-Bericht durchzuschauen.\\
Denke daran, dass der Bericht, dem einer ausgebildeten Fachkraft entsprechen muss.

\begin{taskitemwithoutcomment}{Inhalte im Bericht beurteilen}
  \textbf{Du musst keine Bewertung auf PKOrg vornehmen.}\\
  Nur der Hauptexperte und die verantwortliche Fachkraft tragen ihre Bewertung auf PKOrg ein.\\
  Du findest das Kriterienraster im PDF der Aufgabenstellung im Dokumentenpool der IPA des Kandidaten.
  Es geht um den Teil 1 und 2, die du anhand des abgegebenen Berichts beurteilen kannst.\\
  Mach dir direkt im Bericht und/oder in im Kriterienraster (PDF) Notizen und halte fest, was dir aufgefallen ist.\\
  Lese die Bewertung der verantwortlichen Fachkraft und des Hauptexperten durch und schaue, ob ihr derselben Meinung seid.\\
  Differenzen werden beim 3. Besuch diskutiert.
\end{taskitemwithoutcomment}
\begin{taskitemwithoutcomment}{Mithilfe bei der Vorbereitung des Fachgesprächs}
  Der Hauptexperte ist verantwortlich für die Vorbereitung des Fachgesprächs.\\
  Er wird 6-7 Gesprächsthemen für das Fachgespräch vorbereiten.\\
  Für die Vorbereitung der Fachfragen gibt es eine Vorlage auf PKOrg.\\
  Das Fachgespräch muss aufzeigen, ob der Kandidat in seiner Vertiefungsrichtung kompetent Auskunft geben kann.\\
  Es muss einen Bezug zur IPA haben und darf nicht zu einer reinen Wissensprüfung werden.\\
  Du kannst dich auch einbringen. Vorschläge für Fragen machen. Nimm dazu Kontakt mit dem HEX auf.
\end{taskitemwithoutcomment}
\begin{taskitemwithoutcomment}{Fachthemen aneignen}
  Informiere dich falls nötig zusätzlich zum Thema der IPA. 
  Besonders dann, wenn du mit Teilen der IPA zu wenig vertraut bist, um \enquote{fachsimpeln} zu können.
\end{taskitemwithoutcomment}
\begin{taskitem}{Termin 3. Besuch vereinbaren}
  Der HEX wird den Termin für den 3. Besuch koordinieren.\\
  Sprich ihn darauf an, wenn dir der Termin nicht bekannt ist.\\
\end{taskitem}
\newpage
\begin{taskitem}{Termin 3. Besuch vorbereiten}
  Plane Deine Anreise, damit du pünktlich dort sein kannst.\\
  Kläre mit dem Hauptexperten wie und wo du ihn am besten unterstützen kannst.\\
  Nimm deinen Laptop mit, damit du Notizen machen kannst.\\
  Lege den Kriterienkatalog bereit.\\
  Lege die vom Hauptexperten vorbereiten Fachfragen bereit.
\end{taskitem}
