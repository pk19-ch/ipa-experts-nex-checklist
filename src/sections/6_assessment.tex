\section{Bewertung}
Die Bewertung muss unmittelbar im Anschluss an das Fachgespräch stattfinden.\\
An der Sitzung nehmen nur noch die verantwortliche Fachkraft und die Experten teil.\\
In der Regel wird die Bewertung eine gute Stunde dauern.\\
Der Hauptexperte leitet das Gespräch.\\
Das Ziel ist es, eine Note zu finden, mit welcher die Experten und die verantwortliche Fachkraft einverstanden sind.\\
Bringe dich aktive mit ein.
\begin{taskitem}{Rechtzeitige Abgabe des Berichts}
  Stelle mit dem HEX fest, ob der Bericht rechtzeitig hochgeladen wurde. Ansonsten gibt es einen Notenabzug nach Vorgaben der Prüfungsleitung.
\end{taskitem}
\begin{taskitemwithoutcomment}{Fachgespräch auswerten}
  Es macht Sinn mit der Bewertung des Fachgesprächs zu beginnen.\\
  Die effektiv geführten Dialoge des Fachgesprächs sind zu gruppieren und zu den vorgesehenen Fachgesprächskriterien zuzuordnen.\\
  Spontane Fragen bedingen unter Umständen eine Anpassung der vorbereiteten Kriterien.
  Ergänze die Notizen direkt im Formular zu den Fachfragen.
\end{taskitemwithoutcomment}
\begin{taskitemwithoutcomment}{Bewertung vergleichen und konsolidieren}
  Anschliessend besprecht ihr die Bewertung.\\
  Fokussiere dich auf unterschiedliche Bewertungen zwischen verantwortlicher Fachkraft und Hauptexperte.\\
  Vermeide Grundsatzdiskussionen und ausschweifendes Fachsimpeln.
\end{taskitemwithoutcomment}
\begin{taskitemwithoutcomment}{Bewertungsraster auf PkOrg ergänzen}
  Der Hauptexperte wird laufend das Bewertungsraster im PKOrg nachführen.\\
  Beachte, dass für jedes Kriterium eine nachvollziehbare Begründung eingetragen werden muss.\\
  (Bsp. \enquote{unvollständig}: was genau fehlt oder ist nicht korrekt?)\\
\end{taskitemwithoutcomment}
\begin{taskitem}{Einigkeit}
  Der Notenvorschlag erscheint im PKOrg so bald alle Gütestufen und Bemerkungen eingegeben sind - entspricht das Resultat den Erwartungen?\\
  Bei Uneinigkeit kann die VF dies beim Signieren vermerken. Dann entscheidet die Notenkonferenz.\\
  Bitte diese Möglichkeit nur benutzen, wenn es wirklich nicht anders geht.
\end{taskitem}
\begin{taskitemwithoutcomment}{Signatur der Bewertung}
  Die Bewertung muss von HEX, NEX und VF signiert werden.
\end{taskitemwithoutcomment}
\begin{taskitemwithoutcomment}{Unterlagen und Notizen auf PKOrg hochladen}
  Lade alle deine Notizen in den Dokumentenpool des Kandidaten auf PKOrg hoch.\\
  Achte darauf, dass das die anderen auch tun.
\end{taskitemwithoutcomment}
\begin{taskitemwithoutcomment}{Notenvorschlag}
  Der Hauptexperte muss die verantwortliche Fachkraft darauf aufmerksam machen, dass sie den Notenvorschlag dem Kandidaten nicht mitteilen darf, weil dieser auch im Nachhinein (beim Quervergleich) durch die Prüfungsleitung verändert werden kann.
\end{taskitemwithoutcomment}
\begin{taskitemwithoutcomment}{Vertraulichkeit}
  Behandle alle Informationen zur IPA und alle Unterlagen vertraulich.
\end{taskitemwithoutcomment}
\begin{taskitemwithoutcomment}{Danksagung}
  Bedanke dich bei der verantwortlichen Fachkraft für ihren Einsatz und ermuntere sie, ebenfalls als Experte mitzuwirken. Weitere Informationen dazu können auf \href{https://pk19.ch}{der Webseite der Prüfungsorganisation} gefunden werden.
\end{taskitemwithoutcomment}
\newpage
\begin{taskitem}{Austausch mit Hauptexperten}
  Nach Verabschiedung der verantwortlichen Fachkraft nutze die Gelegenheit und besprechen den Ablauf des Prüfungstages kritisch.\\
  Beispielfragen:\\
  Wie war das Verhältnis zum Kandidaten?\\
  Wie war der Führungsstil?\\
  Gab es Situationen, in der der Kandidat sich unwohl fühlte?\\
  War das Fachgespräch flüssig und angemessen?\\
  Wo hätte man mehr Aufmerksamkeit schenken können? Strenger/weniger streng reagieren?
\end{taskitem}
